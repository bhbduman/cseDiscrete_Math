\documentclass[a4 paper]{article}
\usepackage[inner=2.0cm,outer=2.0cm,top=2.5cm,bottom=2.5cm]{geometry}
\usepackage{setspace}
\usepackage[ruled]{algorithm2e}
\usepackage[rgb]{xcolor}
\usepackage{verbatim}
\usepackage{subcaption}
\usepackage{amsgen,amsmath,amstext,amsbsy,amsopn,tikz,amssymb,tkz-linknodes}
\usepackage{fancyhdr}
\usepackage[colorlinks=true, urlcolor=blue,  linkcolor=blue, citecolor=blue]{hyperref}
\usepackage[colorinlistoftodos]{todonotes}
\usepackage{rotating}
\usepackage{booktabs}
\newcommand{\ra}[1]{\renewcommand{\arraystretch}{#1}}

\newtheorem{thm}{Theorem}[section]
\newtheorem{prop}[thm]{Proposition}
\newtheorem{lem}[thm]{Lemma}
\newtheorem{cor}[thm]{Corollary}
\newtheorem{defn}[thm]{Definition}
\newtheorem{rem}[thm]{Remark}
\numberwithin{equation}{section}

\newcommand{\homework}[6]{
   \pagestyle{myheadings}
   \thispagestyle{plain}
   \newpage
   \setcounter{page}{1}
   \noindent
   \begin{center}
   \framebox{
      \vbox{\vspace{2mm}
    \hbox to 6.28in { {\bf CSE 211:~Discrete Mathematics \hfill {\small (#2)}} }
       \vspace{6mm}
       \hbox to 6.28in { {\Large \hfill #1  \hfill} }
       \vspace{6mm}
       \hbox to 6.28in { {\it Instructor: {\rm #3} \hfill Name: {\rm #5} \hfill Student Id: {\rm #6}} \hfill}
       \hbox to 6.28in { {\it Assistant: #4  \hfill #6}}
      \vspace{2mm}}
   }
   \end{center}
   \markboth{#5 -- #1}{#5 -- #1}
   \vspace*{4mm}
}

\newcommand{\problem}[2]{~\\\fbox{\textbf{Problem #1}}\hfill (#2 points)\newline\newline}
\newcommand{\subproblem}[1]{~\newline\textbf{(#1)}}
\newcommand{\D}{\mathcal{D}}
\newcommand{\Hy}{\mathcal{H}}
\newcommand{\VS}{\textrm{VS}}
\newcommand{\solution}{~\newline\textbf{\textit{(Solution)}} }

\newcommand{\bbF}{\mathbb{F}}
\newcommand{\bbX}{\mathbb{X}}
\newcommand{\bI}{\mathbf{I}}
\newcommand{\bX}{\mathbf{X}}
\newcommand{\bY}{\mathbf{Y}}
\newcommand{\bepsilon}{\boldsymbol{\epsilon}}
\newcommand{\balpha}{\boldsymbol{\alpha}}
\newcommand{\bbeta}{\boldsymbol{\beta}}
\newcommand{\0}{\mathbf{0}}


\begin{document}
\homework{Homework \#4}{Due: 24/12/19}{Dr. Zafeirakis Zafeirakopoulos}{Gizem S\"ung\"u, Başak Karakaş}{Baran Hasan Bozduman}{171044036}
\textbf{Course Policy}: Read all the instructions below carefully before you start working on the assignment, and before you make a submission.
\begin{itemize}
\item It is not a group homework. Do not share your answers to anyone in any circumstance. Any cheating means at least -100 for both sides. 
\item Do not take any information from Internet.
\item No late homework will be accepted. 
\item For any questions about the homework, send an email to gizemsungu@gtu.edu.tr
\item Submit your homework into Assignments/Homework4 directory of the CoCalc project CSE211-2019-2020.
\end{itemize}

\problem{1: Nonhomogeneous Linear Recurrence Relations}{15+15=30}
Consider the nonhomogeneous linear recurrence relation $a_n$ = 3$a_{n-1}$ + $2^n$ .\\
\subproblem{a} Show that whether $a_n$ = $-2^{n+1}$ is a solution of the given recurrence relation or not. Show your work step by step.\\
\solution
\newline
$a_{n-1}=-2^n$ for $a_n$ = $-2^{n+1}$ \\
when we place it in the $a_n = 3a_(n-1) +2^n$ relation\\
$a_n =3(-2^n) +2^n$ and we get\\
$a_n = -2^{n+1}$ and it shows the results are same for $a_n$\\
\subproblem{b} Find the solution with $a_0$ = 1.\\
\solution
\newline
$a_n = 3a_(n-1) +2^n$\\\\
$r = 3 +a^p$\\\\
r=3\\\\
$a_n = c_1(3)^n+ a^p$\\\\
lets assume A is the our const value\\\\
$A.2^n =3.A.2^{n-1} +2^n$\\\\
$2.A = 3.A + 2$\\\\
$A = -2$\\\\
$a_n = c_1(3)^n - 2.2^n$\\\\
$a_0 = 1 = c_1 - 2$\\\\
so $c_1 = 3$\\\\
$a_n = 3(3)^n - 2.2^n$\\\\




\problem{2: Linear Recurrence Relations}{35}
Find all solutions of the recurrence relation $a_n$ = 7$a_{n-1}$ - 16$a_{n-2}$ + 12$a_{n-3}$ + n$4^n$ with $a_0$ = -2, $a_1$ = 0, and $a_2$ = 5.\\
\solution
\newline
$r^3 = 7r^2-16r + 12$\\\\
$(r-2)^2(r-3) = 0$\\\\
$r_1=r_2\neq r_3$\\\\
$2=2\neq3$\\\\
$a_n = c_12^n + c_22^nn + c_33^n$\\\\
$a_n = (An B)4^n$\\\\
$a_{n-1} = (A(n-1) B)4^{n}$\\\\
$a_{n-2} = (A(n-2) B)4^{n-2}$\\\\
$a_{n-3} = (A(n-3) B)4^{n-3}$\\\\
if we put them in the relation and divide them by $4^{n-3}$\\\\
$(An + B)4^3 = 7(A(n-1) +B)4^2 -16(A(n-2) +B )4 + 12(A()n-3) +n4^3$\\\\
if we simplify the equation\\\\
$-4An -20A - 64n + 4B = 0$\\\\
$4B-20A + (-4A-64)n =0$\\\\
$A=16 and B =-80$\\\\
$a_{n}^{p} = (16n - 80)4^n$\\\\
$a_n = c_12^n + c_22^nn + c_33^n + (16n - 80)4^n$\\\\
for $a_0$\\\\
$-2 = c_1 + c_3 -80$\\\\
for $a_1$\\\\
$0 = c_12 + c_22 + c_33 + -256$\\\\
for $a_2$\\\\
$5 = c_14 + c_28 + c_39 + -768$\\\\
$c_1= 17 c_2=\frac{39}{2} c_3 = 61$\\\\
$a_n = 17.2^n +39.2^{n-1}n -61.3^n + (16n + 80)4^n$\\\\
\newline


\problem{3: Linear Homogeneous Recurrence Relations }{20+15 = 35}
Consider the linear homogeneous recurrence relation $a_n$ = 2$a_{n-1}$ - 2$a_{n-2}$.
\subproblem{a} Find the characteristic roots of the recurrence relation.\\
\solution
\newline
$r^2= 2r -2$\\\\
$r^2 -2r +2 = 0$\\\\
we can fid the roots by using delta which is $b^2 - 4ac$\\\\
$\Delta =(-2).(-2) -4.1.2 \rightarrow -4$\\\\
$\frac{2\pm 2\sqrt-1}{2}$\\
$1\pm i$\\\\
$a_n = c_1(1+i)^n +c_2(1-i)^n$\\\\
\subproblem{b} Find the solution of the recurrence relation with $a_0$ = 1 and $a_1$ = 2.\\
\solution\\
$a_0=1 = c_1 +c_2$\\\\
$a_1 = 2 = c_1(1+i) +c_2(1-i)$\\\\
if we apply first equation to second equation like\\\\
$1-c_1 =c_2$\\\\
$c1 +c_1i+1-i-c_1+c_1i$\\\\
$2c_1i-i+1=2$\\\\
$c_1= \frac{-i+1}{2}$\\\\
$1=\frac{1-i}{2}+c_2$\\\\
$c_2=\frac{ i+1}{2}$\\\\
so the relation is\\\\
$a_n = \frac{-i+1}{2}(1+i)^n + \frac{ i+1}{2}(1-i)^n$\\



\end{document} 


