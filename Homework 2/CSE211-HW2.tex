\documentclass[a4 paper]{article}
\usepackage[inner=2.0cm,outer=2.0cm,top=2.5cm,bottom=2.5cm]{geometry}
\usepackage{setspace}
\usepackage[ruled]{algorithm2e}
\usepackage[rgb]{xcolor}
\usepackage{verbatim}
\usepackage{subcaption}
\usepackage{amsgen,amsmath,amstext,amsbsy,amsopn,tikz,amssymb,tkz-linknodes}
\usepackage{fancyhdr}
\usepackage[colorlinks=true, urlcolor=blue,  linkcolor=blue, citecolor=blue]{hyperref}
\usepackage[colorinlistoftodos]{todonotes}
\usepackage{rotating}
\usepackage{booktabs}
\newcommand{\ra}[1]{\renewcommand{\arraystretch}{#1}}

\newtheorem{thm}{Theorem}[section]
\newtheorem{prop}[thm]{Proposition}
\newtheorem{lem}[thm]{Lemma}
\newtheorem{cor}[thm]{Corollary}
\newtheorem{defn}[thm]{Definition}
\newtheorem{rem}[thm]{Remark}
\numberwithin{equation}{section}

\newcommand{\homework}[6]{
   \pagestyle{myheadings}
   \thispagestyle{plain}
   \newpage
   \setcounter{page}{1}
   \noindent
   \begin{center}
   \framebox{
      \vbox{\vspace{2mm}
    \hbox to 6.28in { {\bf CSE 211:~Discrete Mathematics \hfill {\small (#2)}} }
       \vspace{6mm}
       \hbox to 6.28in { {\Large \hfill #1  \hfill} }
       \vspace{6mm}
       \hbox to 6.28in { {\it Instructor: {\rm #3} \hfill Name: {\rm #5} \hfill Student Id: {\rm #6}} \hfill}
       \hbox to 6.28in { {\it Assistant: #4  \hfill #6}}
      \vspace{2mm}}
   }
   \end{center}
   \markboth{#5 -- #1}{#5 -- #1}
   \vspace*{4mm}
}

\newcommand{\problem}[2]{~\\\fbox{\textbf{Problem #1}}\hfill (#2 points)\newline\newline}
\newcommand{\subproblem}[1]{~\newline\textbf{(#1)}}
\newcommand{\D}{\mathcal{D}}
\newcommand{\Hy}{\mathcal{H}}
\newcommand{\VS}{\textrm{VS}}
\newcommand{\solution}{~\newline\textbf{\textit{(Solution)}} }

\newcommand{\bbF}{\mathbb{F}}
\newcommand{\bbX}{\mathbb{X}}
\newcommand{\bI}{\mathbf{I}}
\newcommand{\bX}{\mathbf{X}}
\newcommand{\bY}{\mathbf{Y}}
\newcommand{\bepsilon}{\boldsymbol{\epsilon}}
\newcommand{\balpha}{\boldsymbol{\alpha}}
\newcommand{\bbeta}{\boldsymbol{\beta}}
\newcommand{\0}{\mathbf{0}}


\begin{document}
\homework{Homework \#2}{Due: 12/11/19}{Dr. Zafeirakis Zafeirakopoulos}{Gizem S\"ung\"u}{Baran Hasan Bozduman}{171044036}
\textbf{Course Policy}: Read all the instructions below carefully before you start working on the assignment, and before you make a submission.
\begin{itemize}
\item It is not a group homework. Do not share your answers to anyone in any circumstance. Any cheating means at least -100 for both sides. 
\item Do not take any information from Internet.
\item No late homework will be accepted. 
\item For any questions about the homework, send an email to gizemsungu@gtu.edu.tr
\item Submit your homework into Assignments/Homework1 directory of the CoCalc project CSE211-2019-2020.
\end{itemize}

\problem{1: Sets}{2+2+2+2+2=10}
Which of the following sets are equal? Show your work step by step.\newline
\subproblem{a} $\{$t : t is a root of $x^2$ - 6x + 8 = 0$\}$
\newline
\newline
\subproblem{b} $\{$y : y is a real number in the closed interval [2, 3]$\}$
\newline
\newline
\subproblem{c} $\{$4, 2, 5, 4$\}$
\newline
\newline
\subproblem{d} $\{$4, 5, 7, 2$\}$ - $\{$5, 7$\}$
\newline
\newline
\subproblem{e} $\{$q: q is either the number of sides of a rectangle or the number of digits in any integer between 11 and 99$\}$
\newline
\newline
\solution
\newline
$a) x=2$\space$ and $\space$x=4 $\space$ \rightarrow \left \{ 2,4 \right \}$
\newline
b)$y=\left [ 2,3 \right ]$ 
\newline
c)$\{$4, 2, 5, 4$\}$
\newline
d)$\{$4, 2$\}$
\newline
e)$\{$4, 2$\}$
\newline
so a=d=e

\newpage
\problem{2: Cartesian Product of Sets}{15}
Explain why (A $\times$ B) $\times$ (C $\times$ D) and A $\times$ (B $\times$ C) $\times$ D are not the same.
\solution
\newline
Assume that $a\epsilon A$, $b\epsilon B$, $c\epsilon C$, $d\epsilon D$
\newline
(A $\times$ B) $\times$ (C $\times$ D) = $\left \{ (a,b) \right \}$ X $\left \{ (c,d) \right \}$
\newline
=$\left \{ \left \{ (a,c),(a,d),(b,c),(b,d) \right \} \right \}$
\newline
A $\times$ (B $\times$ C) $\times$ D = $\left \{a \right \}$ X $\left \{(b,c)  \right \}$ X $\left \{ d \right \}$
\newline
=$\left \{ (a,b,c),(b,c,d) \right \}$
\newline
\newline
\newline


\problem{3: Cartesian Product of Sets in Algorithms }{25}
Let A, B and C be sets which have different cardinalities. Let (p, q, r) be each triple of A $\times$ B $\times$ C where p $\in$ A, q $\in$ B and r $\in$ C. Design an algorithm which finds all the triples that are satisfying the criteria: p $\leq$ q and q $\geq$ r. Write the pseudo code of the algorithm in your solution.\newline
\newline
For example: Let the set A, B and C be as A = $\{$ 3, 5, 7 $\}$, B = $\{$ 3, 6 $\}$ and C = $\{$ 4, 6, 9 $\}$. Then the output should be : $\{$ (3, 6, 4), (3, 6, 6), (5, 6, 4), (5, 6, 6) $\}$. \newline
\newline
(Note: Assume that you have sets of A, B, C as an input argument.)\newline
\solution
\begin{algorithm}
\SetAlgoLined
\KwIn{The sets of A, B, C}
\While{$i$ $<$ $n$}{
     \eIf{$a_i$ $<$ $b_i$ $\Lambda$ $c_i$ $<$ $b_i$}{
            $p_i$ $\leftarrow$ $a_i$
        \newline
            $q_i$ $\leftarrow$ $b_i$
        \newline
            $r_i$ $\leftarrow$ $c_i$
        \newline    
             $i$ $\leftarrow$ $i+1$
        \newline
             ( $p_i$ , $q_i$ , $r_i$ ) $_i$
        \newline
    }{
        $i$ $\leftarrow$ $i+1$
    }
}
\caption{Pseudo Code of Your Algorithm}
\end{algorithm}


\newpage
\problem{4: Relations}{3+3+3+3+3+3+3=21}
Determine whether the relation R on the set of all integers is reflexive, symmetric, antisymmetric, and/or transitive, where (x, y) $\in$ R if and only if
\subproblem{a} x $\neq$ y.
\solution
\newline
transitive
\newline
symmetric
\newline
\subproblem{b} xy $\geq$ 1. 
\solution
\newline
reflexitive
\newline
transitive
\newline
symmetric
\newline
\subproblem{c} x = y + 1 or x = y - 1.
\solution
\newline
reflexive
\newline
symmetric
\newline
transitive
\newline
\subproblem{d} x is a multiple of y.
\solution
\newline
reflexice
\newline
antisymmetric
\newline
transitive
\newline
\subproblem{e} x and y are both negative or both nonnegative.
\solution
\newline
reflexitive
\newline
transitive
\newline
symmetric
\newline
\subproblem{f} x $\geq$ $y^2$.
\solution
\newline
antisymmetric
\newline
transitive
\newline
\subproblem{g} x = $y^2$.
\solution
\newline
transitive
\newline
antisymmetric
\newline


\problem{5: Functions}{15}
If f and f $\circ$ g are one-to-one, does it follow that g is one-to-one? Justify your answer.
\solution
\newline
we assume that by conradiction
\newline
Let's say g(x)=g(y) which means its one to one
\newline
fog(x)=fog(y)
\newline
that' means fog(x) and fog(y) reaches the same value and it proof its not one to one  
\newline
so we can say g() is also one to one
\newline

\problem{6: Inverse of Functions}{7+7=14}
Let f be the function from $\mathbb{R}$ to $\mathbb{R}$ defined by f(x) = $x^2$. Find
\subproblem{a} $f^{-1}$ ($\{$ x $\vert$ 0 $<$ x $<$ 1 $\}$) 
\solution
\newline
$f(x)=x^{2}$\space$y=x^{2}\newline $so$\space$ $x=y^{2} $$\sqrt{x}=y\newline$
$f^{-1}=\sqrt{x}$
\newline
$0<\sqrt{x}<1$$\space$ $x=(0,1)$

\subproblem{b}$f^{-1}$ ($\{$ x $\vert$ x $>$ 4 $\}$) 
\solution
\newline
$\sqrt{x}>4$\newline
$x>16$\newline$x= (16,\infty)$
\end{document} 


