\documentclass[a4 paper]{article}
\usepackage[inner=2.0cm,outer=2.0cm,top=2.5cm,bottom=2.5cm]{geometry}
\usepackage{setspace}
\usepackage[rgb]{xcolor}
\usepackage{verbatim}
\usepackage{subcaption}
\usepackage{amsgen,amsmath,amstext,amsbsy,amsopn,tikz,amssymb,tkz-linknodes}
\usepackage{fancyhdr}
\usepackage[colorlinks=true, urlcolor=blue,  linkcolor=blue, citecolor=blue]{hyperref}
\usepackage[colorinlistoftodos]{todonotes}
\usepackage{rotating}
\usepackage{booktabs}
\newcommand{\ra}[1]{\renewcommand{\arraystretch}{#1}}

\newtheorem{thm}{Theorem}[section]
\newtheorem{prop}[thm]{Proposition}
\newtheorem{lem}[thm]{Lemma}
\newtheorem{cor}[thm]{Corollary}
\newtheorem{defn}[thm]{Definition}
\newtheorem{rem}[thm]{Remark}
\numberwithin{equation}{section}

\newcommand{\homework}[6]{
   \pagestyle{myheadings}
   \thispagestyle{plain}
   \newpage
   \setcounter{page}{1}
   \noindent
   \begin{center}
   \framebox{
      \vbox{\vspace{2mm}
    \hbox to 6.28in { {\bf CSE 211:~Discrete Mathematics \hfill {\small (#2)}} }
       \vspace{6mm}
       \hbox to 6.28in { {\Large \hfill #1  \hfill} }
       \vspace{6mm}
       \hbox to 6.28in { {\it Instructor: {\rm #3} \hfill Name: {\rm #5} \hfill Student Id: {\rm #6}} \hfill}
       \hbox to 6.28in { {\it Assistant: #4  \hfill #6}}
      \vspace{2mm}}
   }
   \end{center}
   \markboth{#5 -- #1}{#5 -- #1}
   \vspace*{4mm}
}

\newcommand{\problem}[2]{~\\\fbox{\textbf{Problem #1}}\hfill (#2 points)\newline\newline}
\newcommand{\subproblem}[1]{~\newline\textbf{(#1)}}
\newcommand{\D}{\mathcal{D}}
\newcommand{\Hy}{\mathcal{H}}
\newcommand{\VS}{\textrm{VS}}
\newcommand{\solution}{~\newline\textbf{\textit{(Solution)}} }

\newcommand{\bbF}{\mathbb{F}}
\newcommand{\bbX}{\mathbb{X}}
\newcommand{\bI}{\mathbf{I}}
\newcommand{\bX}{\mathbf{X}}
\newcommand{\bY}{\mathbf{Y}}
\newcommand{\bepsilon}{\boldsymbol{\epsilon}}
\newcommand{\balpha}{\boldsymbol{\alpha}}
\newcommand{\bbeta}{\boldsymbol{\beta}}
\newcommand{\0}{\mathbf{0}}


\begin{document}
\homework{Homework \#1}{Due: 27/10/19}{Dr. Zafeirakis Zafeirakopoulos}{Gizem S\"ung\"u}{Baran Hasan Bozduman}{ 171044036 }
\textbf{Course Policy}: Read all the instructions below carefully before you start working on the assignment, and before you make a submission.
\begin{itemize}
\item It is not a group homework. Do not share your answers to anyone in any circumstance. Any cheating means at least -100 for both sides. 
\item Do not take any information from Internet.
\item No late homework will be accepted. 
\item For any questions about the homework, send an email to gizemsungu@gtu.edu.tr
\item Submit your homework into Assignments/Homework1 directory of the CoCalc project CSE211-2019-2020.
\end{itemize}

\problem{1: Conditional Statements}{5+5+5=15}
State the converse, contrapositive, and inverse of each of these conditional statements.


\subproblem{a} If it snows tonight, then I will stay at home.
\solution
%%%%%%REMOVE \newline commands while writing your answer%%%%%
\newline
\newline
\textbf{Converse:}
  If I will stay at home then it snows tonight
\newline
\textbf{Contrapositive:}
  If I will not stay at home tonight then it does not snow tonight
\newline
\textbf{Inverse:}
  If it does not snow tonight then I will not stay at home




\subproblem{b} I go to the beach whenever it is a sunny summer day.
\solution
%%%%%%REMOVE \newline commands while writing your answer%%%%%
\newline
\newline
\textbf{Converse:}
  It is a sunny summer day whenever I go to the beach
\newline
\textbf{Contrapositive:}
  It is not a sunny summer day whenever I do not go to the beach
\newline
\textbf{Inverse:}
  I do not go to the beach whenever it is not a sunny summer day



\subproblem{c} When I stay up late, it is necessary that I sleep until
noon.
\solution
%%%%%%REMOVE \newline commands while writing your answer%%%%%
\newline
\newline
\textbf{Converse:}
  When It is necessary that I sleep until noon, I stay up late 
\newline
\textbf{Contrapositive:}
  When It is not necessary that I sleep until noon, I do not stay up late
\newline
\textbf{Inverse:}
  When I do not stay up late, It is not necessary that I sleep until noon
\newline

\newpage
\problem{2: Truth Tables For Logic Operators}{5+5+5=15}
Construct a truth table for each of the following compound propositions.
\subproblem{a} (p $\oplus$ $\neg$ q)
\solution
%%%%%%REMOVE \newline commands while writing your answer%%%%%
\begin{displaymath}
\Huge
\begin{array}{|c|c|c|c|}
p & q & \neg q & p \oplus \neg q\\ 
\hline
1 & 1 & 0 & 1\\
1 & 0 & 1 & 0\\
0 & 1 & 0 & 0\\
0 & 0 & 1 & 1\\
\end{array}
\end{displaymath}
\newline
\newline
\subproblem{b} (p $\iff$ q) $\oplus$ ( $\neg$ p $\iff$ $\neg$ r)
\solution
\begin{displaymath}
\huge
\begin{array}{|c|c|c|c|c|c|c|c|}
\neg p& p & q & r & \neg r & p \Leftrightarrow q & \neg p \Leftrightarrow \neg r & ( p \Leftrightarrow q ) \oplus ( \neg p \Leftrightarrow \neg r)\\ 
\hline
0 & 1 & 1 & 1 & 0 & 1 & 1 & 0\\
0 & 1 & 1 & 0 & 1 & 1 & 0 & 1\\
0 & 1 & 0 & 1 & 0 & 0 & 1 & 1\\
0 & 1 & 0 & 0 & 1 & 0 & 0 & 0\\
1 & 0 & 1 & 1 & 0 & 0 & 0 & 0\\
1 & 0 & 1 & 0 & 1 & 0 & 1 & 1\\
1 & 0 & 0 & 1 & 0 & 1 & 0 & 1\\
1 & 0 & 0 & 0 & 1 & 1 & 1 & 0\\
\end{array}
\end{displaymath}


\newpage
\subproblem{c} (p $\oplus$ q) $\Rightarrow$ (p $\oplus$ $\neg$ q)
\solution
\begin{displaymath}
\huge
\begin{array}{|c|c|c|c|c|c|}
p & q & \neg q & p \oplus q &  p \oplus \neg q & (p \oplus q )\Rightarrow (p \oplus \neg q) \\ 
\hline
1 & 1 & 0 & 0 & 1 & 1\\
1 & 0 & 1 & 1 & 0 & 0\\
0 & 1 & 0 & 1 & 0 & 0\\
0 & 0 & 1 & 0 & 1 & 1\\
\end{array}
\end{displaymath}






\newpage
\problem{3: Logic in Algorithms }{10+10+10=30}
If x = 1 before the statement is reached, what is the value of x after each of these statements is encountered in a computer program? Why? Show your work step by step.
\subproblem{a}  \textbf{for} i $\Leftarrow$ 1 \textbf{to} 10 \textbf{do}\\ 
\par \hspace{5mm} \textbf{if} x + 2 = 3 \textbf{then} x := x + 1 \newline
\par \hspace{0.5mm} \textbf{end}
\newline
\solution
%%%%%%REMOVE \newline commands while writing your answer%%%%%
\newline
for i=1   x+2=3 is an true statement so we increase x and new x value is 2\\
for i=2 4=3 is false satement \newline for i=3 5=3 is false satement \newline for i=4 6=3 is false satement \newline for i=5 7=3 is false satement \newline for i= 6 8=3 is false satement \newline for i= 7 9=3 is false satement \newline for i=8 10=3 is false satement \newline for i=9 11=3 is false satement \newline for i=10 statement is false\\
so the last x value is still 2\\
\newline
\newline

\subproblem{b}  \textbf{for} i $\Leftarrow$ 1 \textbf{to} 5 \textbf{do}\\ 
\par \hspace{5mm} \textbf{if} (x + 1 = 2) XOR (x + 2 = 3) \textbf{then} x := x + 1
\newline
\par \hspace{0.5mm} \textbf{end}
\newline
\solution
%%%%%%REMOVE \newline commands while writing your answer%%%%%
\newline
for i=1 2=2 XOR 3=3 is false so x=1\\
for i=2 3=2 XOR 4=3 is false so x=1\\
for i=3 4=2 XOR 5=3 is false so x=1\\
for i=4 5=2 XOR 6=3 is false so x=1\\
for i=5 6=2 XOR 7=3 is false so x=1\\
finally the x value is still 1\\
\newline

\subproblem{c}  \textbf{for} i $\Leftarrow$ 1 \textbf{to} 4 \textbf{do} \newline
\par \hspace{5mm} \textbf{if} (2x + 3 = 5) AND (3x + 4 = 7) \textbf{then} x := x + 1
\newline
\par \hspace{0.5mm} \textbf{end}
\newline 
\solution
%%%%%%REMOVE \newline commands while writing your answer%%%%%
\newline
for i=1 5=5 AND 7=7 is a true statement so we increase x 1 new x is x=2\\
for i=2 7=5 AND 10=7 is a false statement so x still 2\\
for i=3 9=5 AND 13=7 is a false statement\\
for i=4 11=5 AND 16=7 is a false statement\\
finally x value is still 2\\
\newline



\newpage
\problem{4: Proof by contradiction }{20}
Show that at least three of any 25 days chosen must fall in the same month of the year using a proof by contradiction. Explain your work step by step.
\solution
%%%%%%REMOVE \newline commands while writing your answer%%%%%
\newline
since inverse of atleast theree days is smaller three days we can express a  number which is smaller than three\\
Let's chose 2 days for each month(to be fewer than at least 3days)\\
There are 12 months in a year \\
12x2=24\\
but the given number of days 25 so we have to chose one month more\\
but we have only 12 month so one of the month has three days\\
finally we have at least one month which has three days\\
\newline
\problem{5: Proof by contraposition }{20}
  Show that if $n^3 + 5$ is odd, then n is even using a proof by contraposition. Explain your work step by step.\\
\textit{Note: Assume that n is an integer.}
\solution\\
the contraposition of statement is "If n is odd then $n^3+5$ is even" so\\
we assume n is odd\\
since n=2k+1 \\
$n^3 +5$ =$(2k+1)^3$ +5\\
$n^3 +5$ =$8k^3$ +$12k^2$ + $6k$ +1 +5\\
$n^3 +5$ =$8k^3$ +$12k^2$ + 6k +6\\
$n^3 +5$ =2$(4k^3 +6k^2 + 3k +3)$\\
let's say $(4k^3 +6k^2 + 3k +3)$ is p\\
$n^3 +5$ =2p\\
2p is even $n^3 +5$ is also even\\
since "If n is odd then $n^3 +5$ is even" is true which is the contraposition of given statement\\
then the original statement is also true. \\
\end{document} 
